% Created 2019-05-26 Sun 12:10
% Intended LaTeX compiler: pdflatex
\documentclass[11pt]{article}
\usepackage[utf8]{inputenc}
\usepackage[T1]{fontenc}
\usepackage{graphicx}
\usepackage{grffile}
\usepackage{longtable}
\usepackage{wrapfig}
\usepackage{rotating}
\usepackage[normalem]{ulem}
\usepackage{amsmath}
\usepackage{textcomp}
\usepackage{amssymb}
\usepackage{capt-of}
\usepackage{hyperref}
\usepackage{minted}
\usepackage{parskip}
\author{Marc-André Chénier}
\date{\today}
\title{}
\hypersetup{
 pdfauthor={Marc-André Chénier},
 pdftitle={},
 pdfkeywords={},
 pdfsubject={},
 pdfcreator={Emacs 26.2 (Org mode 9.1.9)}, 
 pdflang={English}}
\begin{document}

\tableofcontents


\section{Question 1}
\label{sec:org134d2f8}
Binomial likelihood, the posterior with a conjugate beta prior is:
\begin{equation}
p(\theta | y)=\frac{1}{B(\overline{\alpha}, \overline{\beta})} 
\theta^{\overline{\alpha}-1}(1-\theta)^{\overline{\beta}-1}
\end{equation}
with 
\begin{equation}
\begin{aligned} \overline{\alpha} &=\alpha_{0}+y \\ 
\overline{\beta} &=\beta_{0}+n-y \end{aligned}
\end{equation}

The beta prior can be specified as:
\begin{equation}
\equiv \text { binomial experiment with }\left(\alpha_{0}-1\right) 
\text { successes in }\left(\alpha_{0}+\beta_{0}-2\right)
\end{equation}

\section{Question 2}
\label{sec:org41d5da7}
Check PPD for binomial likelihood on p. 151. We should take into
account sampling variability of \(\hat{\theta}\)

\section{Question 3}
\label{sec:orgb6d7cdb}
Contour probability: posterior evidence of \(H_{0}\) with HPD interval.
Defined as:
\begin{equation}
P\left[p(\theta | \boldsymbol{y})>p\left(\theta_{0} | 
\boldsymbol{y}\right)\right] \equiv\left(1-p_{B}\right)
\end{equation}

\(p_{B}\) is computed from the smallest HPD interval containing
\(\theta_{0}\). 

\(\operatorname{Beta}\left(\alpha_{0}, \beta_{0}\right)\) prior is
equivalent to a binomial experiment with \(\alpha_{0} - 1\) successes in
(\(\alpha_{0} + \beta_{0} - 2\)) experiments.

The non-informative beta prior has \(\alpha_{0}=1, \beta_{0}=1\) and is
equal the uniform prior on \([0, 1]\).


\section{Question 4}
\label{sec:org8e86954}
Popular priors for BGLIM are normal proper priors with large variance.
Gelman et al. however suggest Cauchy density with center 0 and scale
parameter 2.5 for standardized continuous covariates.
\end{document}